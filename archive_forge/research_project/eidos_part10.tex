\documentclass[11pt]{article}
\usepackage{amsmath, amssymb, amsthm, mathtools, enumitem, geometry, hyperref, algorithm, algpseudocode}
\geometry{letterpaper, margin=1in}
\hypersetup{colorlinks=true, linkcolor=blue, citecolor=blue, urlcolor=blue}

\title{Module J: Recursive Adaptive Dynamic Idempotent Feedback and State-Based Runtime Learning and Inference \\ 
\large Part of the Eidos Unified Framework for Persistent, Dynamic, and Adaptive Multimodal Intelligence}
\author{---}
\date{}

\begin{document}

\maketitle

\tableofcontents
\newpage

%%%%%%%%%%%%%%%%%%%%%%%%%%%%%%%%%%%%%%%%%%%%%%%%%%%%%%%%%%%%%%%%%%%%%%
\section{Abstract}
%%%%%%%%%%%%%%%%%%%%%%%%%%%%%%%%%%%%%%%%%%%%%%%%%%%%%%%%%%%%%%%%%%%%%%
This module rigorously defines the \emph{Recursive Adaptive Dynamic Idempotent Feedback and State-Based Runtime Learning and Inference} component of the Eidos framework. It provides a formal system for modeling interdependent feedback dynamics among multiple entities, capturing how triggers and actions propagate influence recursively. Key aspects include the definition of entities, triggers, actions, influence functions, and feedback loops; the introduction of adaptive modulation functions that update system states based on external and internal signals; and the enforcement of idempotence and reversibility to guarantee stability. The framework is developed with full mathematical rigor and is designed to be modular, extensible, and dynamically adaptive, enabling continuous runtime learning and inference.

%%%%%%%%%%%%%%%%%%%%%%%%%%%%%%%%%%%%%%%%%%%%%%%%%%%%%%%%%%%%%%%%%%%%%%
\section{Introduction and Motivation}
%%%%%%%%%%%%%%%%%%%%%%%%%%%%%%%%%%%%%%%%%%%%%%%%%%%%%%%%%%%%%%%%%%%%%%
In complex intelligent systems, continuous adaptation and learning require mechanisms that can recursively integrate feedback from multiple sources. The \emph{Recursive Adaptive Dynamic Idempotent Feedback} system in the Eidos framework addresses this need by modeling the interactions among various entities through triggers and actions. These interactions are recursively composed into feedback loops that update the system state in an adaptive yet stable manner. Key motivations for this module include:
\begin{enumerate}[label=(\alph*)]
    \item Modeling the bidirectional and cyclic interactions among entities.
    \item Ensuring that repeated updates via feedback are idempotent, thereby guaranteeing stability.
    \item Allowing dynamic adaptation at runtime through adaptive modulation of entity states.
    \item Supporting a modular design where each entity’s feedback can be updated independently without disrupting the global system state.
\end{enumerate}
This module provides the formal and algorithmic foundation for runtime learning and inference based on recursive feedback.

%%%%%%%%%%%%%%%%%%%%%%%%%%%%%%%%%%%%%%%%%%%%%%%%%%%%%%%%%%%%%%%%%%%%%%
\section{Preliminaries and Notation}
%%%%%%%%%%%%%%%%%%%%%%%%%%%%%%%%%%%%%%%%%%%%%%%%%%%%%%%%%%%%%%%%%%%%%%
We introduce the following notation and definitions:

\begin{itemize}[label=\(\bullet\)]
    \item Let \(\mathcal{E} = \{E_i \mid i \in I\}\) denote the set of entities in the system, where \(I\) is an index set (finite or countable).
    \item For each entity \(E_i\):
    \begin{itemize}[label=\(\circ\)]
        \item \(\tau(E_i) \in \mathcal{T}\) represents the \emph{trigger} function or signal.
        \item \(\alpha(E_i) \in \mathcal{A}\) represents the \emph{action} executed by the entity.
    \end{itemize}
    \item The \emph{influence function} is denoted by:
    \[
      \delta(E_i, E_j): \quad \tau(E_i) \longrightarrow \alpha(E_j),
    \]
    which quantifies how the trigger of \(E_i\) affects the action of \(E_j\).
    \item The \emph{feedback function} is defined as:
    \[
      \phi(E_i, E_j) = \delta(E_i, E_j) \circ \delta(E_j, E_i),
    \]
    where \(\circ\) denotes functional composition, capturing the bidirectional recursive interaction.
    \item The \emph{system state} for an entity \(E_i\) is denoted by:
    \[
      \Sigma(E_i) = \{\tau(E_i),\, \alpha(E_i),\, \{\delta(E_i, E_j)\}_{j \neq i}\},
    \]
    and for a set of entities relevant to a context \(a \in \mathcal{C}\), the global state is:
    \[
      \Sigma_a = \bigoplus_{(E_i, E_j) \in \mathcal{E}_a^2,\; i\neq j} \left[\phi(E_i, E_j) \oplus \phi(E_j, E_i)\right],
    \]
    where \(\oplus\) denotes a modular composition operator (assumed to be associative and commutative).
    \item An \emph{adaptive modulation function} is defined as:
    \[
      \mu_a: \Sigma_0 \to \Sigma_a,
    \]
    which adapts a base state \(\Sigma_0\) to a given context \(a\). This function is required to be idempotent:
    \[
      \mu_a\bigl(\mu_a(\Sigma_0)\bigr) = \mu_a(\Sigma_0).
    \]
    \item The system may incorporate adaptive parameters from a set \(\mathcal{P}\), with the learned modulation for context \(a\) represented as:
    \[
      \mathcal{P}_a \in \mathcal{P}.
    \]
\end{itemize}

%%%%%%%%%%%%%%%%%%%%%%%%%%%%%%%%%%%%%%%%%%%%%%%%%%%%%%%%%%%%%%%%%%%%%%
\section{Formal Definitions and Mathematical Formulation}
%%%%%%%%%%%%%%%%%%%%%%%%%%%%%%%%%%%%%%%%%%%%%%%%%%%%%%%%%%%%%%%%%%%%%%

\subsection*{Definition J.1 (Entities, Triggers, and Actions)}
For each entity \(E_i \in \mathcal{E}\):
\[
\tau(E_i) \in \mathcal{T}, \quad \alpha(E_i) \in \mathcal{A}.
\]
These functions are assumed to be defined on appropriate domains and codomains so that they can be composed with influence functions.

\subsection*{Definition J.2 (Influence Function)}
For any two distinct entities \(E_i, E_j \in \mathcal{E}\), the influence function is defined as:
\[
\delta(E_i, E_j): \quad \tau(E_i) \longrightarrow \alpha(E_j).
\]
This function quantifies the manner in which the trigger signal of \(E_i\) induces or modulates the action of \(E_j\).

\subsection*{Definition J.3 (Feedback Function)}
The bidirectional feedback between entities \(E_i\) and \(E_j\) is defined as:
\[
\phi(E_i, E_j) = \delta(E_i, E_j) \circ \delta(E_j, E_i).
\]
This composition represents the recursive interplay between \(E_i\) and \(E_j\), where the output of one influence serves as the input to the other. A non-trivial feedback loop satisfies:
\[
\phi(E_i, E_j) \neq 0,
\]
ensuring meaningful interaction.

\subsection*{Definition J.4 (System State)}
For a given context \(a \in \mathcal{C}\), define the pairwise state contribution of entities \(E_i\) and \(E_j\) as:
\[
\Sigma_a(E_i, E_j) = \phi(E_i, E_j) \oplus \phi(E_j, E_i).
\]
The overall system state under context \(a\) is then:
\[
\Sigma_a = \bigoplus_{(E_i, E_j) \in \mathcal{E}_a^2,\; i\neq j} \Sigma_a(E_i, E_j).
\]

\subsection*{Definition J.5 (Adaptive Modulation Function)}
Define a modulation function that adapts the base state \(\Sigma_0\) to a context \(a\):
\[
\mu_a: \Sigma_0 \to \Sigma_a,
\]
with the idempotence property:
\[
\mu_a\bigl(\mu_a(\Sigma_0)\bigr) = \mu_a(\Sigma_0).
\]
When the modulation function is learned or trained, it encapsulates a set of adaptive parameters:
\[
\mathcal{P}_a \in \mathcal{P},
\]
which are then applied to update the system state.

%%%%%%%%%%%%%%%%%%%%%%%%%%%%%%%%%%%%%%%%%%%%%%%%%%%%%%%%%%%%%%%%%%%%%%
\section{Algorithmic Description}
%%%%%%%%%%%%%%%%%%%%%%%%%%%%%%%%%%%%%%%%%%%%%%%%%%%%%%%%%%%%%%%%%%%%%%

The following pseudocode outlines the operation of the recursive feedback system:

\begin{algorithm}[H]
\caption{Recursive Adaptive Feedback and Runtime Learning}
\label{alg:recursive_feedback}
\begin{algorithmic}[1]
    \State \textbf{Input:} Set of entities \( \mathcal{E} = \{E_i \mid i \in I\} \); initial system state \(\Sigma_0\); context \( a \in \mathcal{C} \); modulation function \(\mu_a\)
    \State \textbf{Output:} Adapted system state \(\Sigma_a\)
    \State \textbf{Begin:}
    \For{each pair of distinct entities \( (E_i, E_j) \)}
        \State Compute influence: \( \delta(E_i, E_j) \) from \(\tau(E_i)\) to \(\alpha(E_j)\)
        \State Compute reverse influence: \( \delta(E_j, E_i) \) from \(\tau(E_j)\) to \(\alpha(E_i)\)
        \State Compute feedback: 
        \[
          \phi(E_i, E_j) \gets \delta(E_i, E_j) \circ \delta(E_j, E_i)
        \]
        \State Set pairwise state: 
        \[
          \Sigma_a(E_i, E_j) \gets \phi(E_i, E_j) \oplus \phi(E_j, E_i)
        \]
    \EndFor
    \State Compose global state: 
    \[
      \Sigma_a \gets \bigoplus_{(E_i,E_j) \in \mathcal{E}_a^2,\; i\neq j} \Sigma_a(E_i, E_j)
    \]
    \State Apply modulation: 
    \[
      \Sigma_a \gets \mu_a(\Sigma_0)
    \]
    \State \textbf{Return:} \(\Sigma_a\)
\end{algorithmic}
\end{algorithm}

\noindent \textbf{Optional Iterative Adaptation:}  
For continual runtime learning, the above procedure can be applied iteratively. Let:
\[
\Sigma^{(0)} = \Sigma_0,\quad \Sigma^{(t+1)} = \mu_a\Bigl(\Sigma^{(t)}\Bigr),
\]
with convergence guaranteed by the idempotence property:
\[
\lim_{t \to \infty} \Sigma^{(t)} = \Sigma_a.
\]

%%%%%%%%%%%%%%%%%%%%%%%%%%%%%%%%%%%%%%%%%%%%%%%%%%%%%%%%%%%%%%%%%%%%%%
\section{Theoretical Analysis and Guarantees}
%%%%%%%%%%%%%%%%%%%%%%%%%%%%%%%%%%%%%%%%%%%%%%%%%%%%%%%%%%%%%%%%%%%%%%

\subsection*{Theorem J.1 (Idempotence and Stability)}
\textbf{Statement:}  
For any context \( a \in \mathcal{C} \), the modulation function \( \mu_a \) is idempotent, i.e.,
\[
\mu_a\bigl(\mu_a(\Sigma_0)\bigr) = \mu_a(\Sigma_0) = \Sigma_a.
\]
Thus, repeated application of the feedback update yields a stable system state.

\textbf{Proof Sketch:}  
By the definition of \(\mu_a\) (Definition J.5), once the base state \(\Sigma_0\) is adapted to \(\Sigma_a\), further applications do not alter \(\Sigma_a\). This follows from the property that the learned adaptive parameters \(\mathcal{P}_a\) yield an idempotent mapping. \(\Box\)

\subsection*{Proposition J.2 (Modularity and Scalability)}
The feedback system is constructed via the modular composition operator \(\oplus\), which is assumed to be associative and commutative. Consequently, the global state
\[
\Sigma_a = \bigoplus_{(E_i,E_j) \in \mathcal{E}_a^2,\; i\neq j} \Sigma_a(E_i, E_j)
\]
is independent of the order of composition. This property ensures that the system can scale to an arbitrary number of entities and can be extended to higher-order interactions without loss of consistency.

%%%%%%%%%%%%%%%%%%%%%%%%%%%%%%%%%%%%%%%%%%%%%%%%%%%%%%%%%%%%%%%%%%%%%%
\section{Integration with the Overall Eidos Framework}
%%%%%%%%%%%%%%%%%%%%%%%%%%%%%%%%%%%%%%%%%%%%%%%%%%%%%%%%%%%%%%%%%%%%%%

Module J is a core component responsible for enabling adaptive, runtime learning and inference. It:
\begin{itemize}[label=\(\bullet\)]
    \item Interfaces with the deep model architectures (Module H) by providing a mechanism for continuous state adaptation based on internal feedback.
    \item Interacts with the memory module (Module I) by incorporating external, real-time signals into the feedback loops.
    \item Connects with the training system (Module K) to allow online parameter updates and continual learning.
    \item Provides a recursive framework that ensures that system states remain stable (via idempotence) while adapting dynamically to new inputs and contextual changes.
\end{itemize}

%%%%%%%%%%%%%%%%%%%%%%%%%%%%%%%%%%%%%%%%%%%%%%%%%%%%%%%%%%%%%%%%%%%%%%
\section{Implementation Considerations}
%%%%%%%%%%%%%%%%%%%%%%%%%%%%%%%%%%%%%%%%%%%%%%%%%%%%%%%%%%%%%%%%%%%%%%

\begin{itemize}[label=\(\bullet\)]
    \item \textbf{Feedback Computation:}  
    The functions \(\delta(E_i,E_j)\) should be implemented using well-established techniques (e.g., weighted mappings or neural network modules) with careful calibration to avoid vanishing or exploding feedback signals.
    \item \textbf{Adaptive Modulation:}  
    The modulation function \(\mu_a\) may be implemented as a gating network or a parameterized transformation that is trained jointly with the rest of the model.
    \item \textbf{Idempotence Verification:}  
    Empirical tests should be designed to verify that repeated applications of \(\mu_a\) yield negligible changes beyond a certain iteration, ensuring convergence.
    \item \textbf{Scalability:}  
    The composition operator \(\oplus\) should be chosen to support efficient aggregation over large sets of entities. Sparse representations and parallel computation may be used.
    \item \textbf{Integration with Feedback Sources:}  
    The system should accommodate both internally generated signals (e.g., prediction errors) and external signals (e.g., user feedback) within the recursive feedback loops.
\end{itemize}

%%%%%%%%%%%%%%%%%%%%%%%%%%%%%%%%%%%%%%%%%%%%%%%%%%%%%%%%%%%%%%%%%%%%%%
\section{Conclusion}
%%%%%%%%%%%%%%%%%%%%%%%%%%%%%%%%%%%%%%%%%%%%%%%%%%%%%%%%%%%%%%%%%%%%%%
In this module, we have developed a comprehensive framework for recursive, adaptive, and idempotent feedback that underpins runtime learning and inference in the Eidos framework. The key contributions are:
\begin{itemize}[label=\(\bullet\)]
    \item A formal definition of entities, triggers, actions, and influence functions that model interdependent interactions.
    \item The definition of a recursive feedback function \(\phi(E_i,E_j)\) that composes bidirectional influences.
    \item The construction of a global system state \(\Sigma_a\) via modular composition, capturing the cumulative effects of feedback across entities.
    \item The introduction of an adaptive modulation function \(\mu_a\) with an idempotence property, ensuring that the system state converges and remains stable under repeated updates.
    \item The presentation of algorithmic procedures for computing feedback and adapting system parameters in real time.
\end{itemize}
This robust and modular recursive feedback mechanism is essential for enabling dynamic adaptation, continual learning, and runtime inference in complex environments, forming a critical component of the overall Eidos framework.

%%%%%%%%%%%%%%%%%%%%%%%%%%%%%%%%%%%%%%%%%%%%%%%%%%%%%%%%%%%%%%%%%%%%%%
\section{Module Summary}
%%%%%%%%%%%%%%%%%%%%%%%%%%%%%%%%%%%%%%%%%%%%%%%%%%%%%%%%%%%%%%%%%%%%%%
\textbf{Completed:}
\begin{itemize}[label=\(\bullet\)]
    \item Module A: Input Processing.
    \item Module B: Universal Communication \& Data Handling Interface and Coordination.
    \item Module C: Universal Streaming/Handling/Loading/Indexing Module.
    \item Module D: Multidimensional Vocabulary and Tokenization System.
    \item Module E: Contextual NLU/NLP Embedding and Multidimensional Tokenization.
    \item Module F: Deep Knowledge Graphs System (Base and Personal).
    \item Module G: Infinite RoPE Context Scaling and Dynamic Vocabulary Updating.
    \item Module H: Core Model Architectures (RWKV and Transformer Modules, Mixture-of-Experts Style).
    \item Module I: Titans Memory Architecture (Multi-Layer Memory Module).
    \item Module J: Recursive Adaptive Dynamic Idempotent Feedback and State-Based Runtime Learning and Inference.
\end{itemize}
\textbf{Remaining Modules:}
\begin{itemize}[label=\(\bullet\)]
    \item Module K: Universal Training System.
    \item Module L: Final Decoding and Multimodal Output.
\end{itemize}

\end{document}
