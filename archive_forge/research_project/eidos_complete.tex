%%%%%%%%%%%%%%%%%%%%%%%%%%%%%%%%%%%%%%%%%%%%%%%%%%%%%%%%%%%%%%%%%%%%%%
% Begin perfected file: eidos_complete.tex
%%%%%%%%%%%%%%%%%%%%%%%%%%%%%%%%%%%%%%%%%%%%%%%%%%%%%%%%%%%%%%%%%%%%%%

\documentclass[11pt]{article}
\usepackage[utf8]{inputenc}
\usepackage[T1]{fontenc}
\usepackage{amsmath,amssymb,amsthm,mathtools,enumitem,geometry,hyperref,physics,cleveref,thmtools,mathrsfs,amsopn}
\usepackage{algorithm}
\usepackage{algpseudocode}
\usepackage{tikz-cd} % for commutative diagram
\geometry{letterpaper, margin=1in}
\hypersetup{
  colorlinks   = true,
  linkcolor    = blue,
  citecolor    = blue,
  urlcolor     = blue
}
\DeclareMathOperator*{\argmin}{arg\,min}
\DeclareMathOperator{\diag}{diag}
\newcommand{\R}{\mathbb{R}}
\newcommand{\N}{\mathbb{N}}
\newcommand{\C}{\mathbb{C}}
\newcommand{\Xproc}{\mathcal{X}_{\mathrm{proc}}}

\title{Eidos: A Unified Framework for Persistent, Dynamic, and Adaptive Multimodal Intelligence}
\author{---}
\date{}

\begin{document}

\maketitle

\begin{abstract}
We introduce \emph{Eidos}---an avant-garde, unified framework that seamlessly integrates raw input processing; universal communication, data handling, and streaming infrastructures; a multidimensional vocabulary and tokenization schema; dual-layer (base and adaptive) embeddings supporting both natural language understanding (NLU) and natural language processing (NLP); hierarchically organized knowledge graphs (base, personal, and unified); infinite context scaling via Rotary Positional Embeddings (RoPE) coupled with dynamic vocabulary refinement; a hybrid mixture-of-experts core model uniting Transformer, RWKV, and related modules; a multilayer "Titans" memory architecture (encompassing short-term, working, long-term, and personal memory); recursive adaptive idempotent feedback for continuous runtime learning; and an all-encompassing universal training system. The framework decodes outputs (text by default, with scalable pathways for additional modalities) in a fully modular, scalable, and extensible manner. Every symbol, process, function, and interface is meticulously defined using distinct notation and precise algorithmic pseudocode, constituting a robust blueprint for empirical implementation, testing, and iterative refinement.
\end{abstract}

\tableofcontents
\newpage

%%%%%%%%%%%%%%%%%%%%%%%%%%%%%%%%%%%%%%%%%%%%%%%%%%%%%%%%%%%%%%%%%%%%%%
\section{Notational Conventions and Distinct Symbol Sets}
%%%%%%%%%%%%%%%%%%%%%%%%%%%%%%%%%%%%%%%%%%%%%%%%%%%%%%%%%%%%%%%%%%%%%%

\subsection*{Fundamental Axioms}
The Eidos framework operates under three core axioms:
\begin{enumerate}[label=(\textbf{A\arabic*}),leftmargin=*]
    \item \textbf{Persistent Adaptivity}: $\forall t\,\exists\,\Delta\theta_t\in\Theta_{\mathrm{adapt}}$ such that 
    \[
      f_{\theta_{t+1}}=f_{\theta_t}\oplus\Delta\theta_t.
    \]
    \item \textbf{Idempotent Recursion}: 
    \[
      \mu^{\mathrm{R}}\circ\mu^{\mathrm{R}}\equiv\mu^{\mathrm{R}}.
    \]
    \item \textbf{Unified Multimodality}: 
    \[
      \mathcal{O}_{\mathrm{mod}}=\bigoplus_{m\in\mathcal{M}}\delta_m(\mathcal{Y}).
    \]
\end{enumerate}

\subsection{Raw Input and Preprocessing}

\begin{itemize}[label=\(\bullet\)]
  \item \textbf{Raw Input:} $X_{\mathrm{raw}}\in\Sigma^*$, where $\Sigma$ denotes the base alphabet (e.g., Unicode).
  \item \textbf{Preprocessed Input:} $X_{\mathrm{proc}}\in\mathcal{X}_{\mathrm{proc}}$.
  \item \textbf{Preprocessing Operator:} $\mathcal{P}_{\mathrm{in}}:\Sigma^*\to\mathcal{X}_{\mathrm{proc}}$.
\end{itemize}

\subsection{Vocabulary and Tokens}

\begin{itemize}[label=\(\bullet\)]
  \item \textbf{Base Vocabulary:} $\mathcal{V}^{(0)}$ (e.g., English words, Unicode characters, programming symbols).
  \item \textbf{Learned Tokens:} $\mathcal{V}^{(1)}$ (multi-token sequences).
  \item \textbf{Complete Vocabulary:} $\mathcal{V}=\mathcal{V}^{(0)}\cup\mathcal{V}^{(1)}$.
  \item \textbf{Token:} $t\in\mathcal{V}$.
  \item \textbf{Unique Identifier Mapping:} $\eta:\mathcal{V}\to\mathbb{N}$, with $\operatorname{ID}(t)=\eta(t)$.
  \item \textbf{Token Structure:} Each token is represented as
  \[
    t=\bigl(u,\,\pi,\,\chi\bigr),
  \]
  where
  \begin{itemize}[label=\(\circ\)]
    \item $u$: the underlying unit (string),
    \item $\pi\in\Pi\subseteq\mathbb{R}^{d_{\pi}}$: intrinsic properties,
    \item $\chi\in\mathbb{R}^{d_{\chi}}$: contextual statistics.
  \end{itemize}
\end{itemize}

\subsection{Enhanced Core Definitions}
In addition to the standard representations, Eidos incorporates advanced mathematical constructs to capture quantum-like behavior and complex dynamics:

\subsubsection*{Quantum Token Representation}
Each token exists in a superposition state:
\[
\ket{t}=\alpha\ket{u}\otimes\beta\ket{\pi}\otimes\gamma\ket{\chi}\quad\text{where }|\alpha|^2+|\beta|^2+|\gamma|^2=1,
\]
with measurement operators $\mathcal{M}_{\mathrm{ctx}}=\{\Pi_{\mathrm{base}},\Pi_{\mathrm{pers}}\}$.

\subsubsection*{Holomorphic Knowledge Embedding}
Embeddings evolve via Cauchy--Riemann dynamics:
\[
\frac{\partial E}{\partial\overline{z}}=0\quad\text{with }z=x+i\xi\in\mathbb{C}^{d_E\times d_{\mathrm{pers}}},
\]
ensuring analyticity in the joint embedding space.

\subsubsection*{Noncommutative Memory Operators}
Memory operations form a C*-algebra:
\[
[\mathcal{M}_i,\mathcal{M}_j]=i\epsilon_{ijk}\mathcal{M}_k\quad\text{with }\mathcal{M}\in\mathfrak{su}(d_{\mathcal{M}}),
\]
capturing the inherent noncommutativity in dynamic memory updates.

\subsection{Embeddings}

\begin{itemize}[label=\(\bullet\)]
  \item \textbf{Base Embedding:} $E_{\mathrm{B}}:\mathcal{V}\to\mathbb{R}^{d_E}$.
  \item \textbf{Contextual Embedding:} $E_{\mathrm{C}}:(\mathbb{R}^{d_E})^n\to(\mathbb{R}^{d_C})^n$, for a sequence of length $n$.
  \item \textbf{Fusion Operator:} $g:\mathbb{R}^{d_E}\times\mathbb{R}^{d_C}\to\mathbb{R}^{d_F}$.
  \item \textbf{Final Token Representation:} $\mathbf{z}_i=E_{\mathrm{F}}(t_i,\xi)\in\mathbb{R}^{d_F}$, where $\xi$ denotes contextual or personalized parameters.
\end{itemize}

\subsection{Knowledge Graphs}

\begin{itemize}[label=\(\bullet\)]
  \item \textbf{Base Knowledge Graph (BKG):} $\mathcal{G}_{\mathrm{BKG}}=(\mathcal{N}_{\mathrm{BKG}},\mathcal{E}_{\mathrm{BKG}})$, with nodes defined by $E_{\mathrm{B}}(t)$.
  \item \textbf{Personal Knowledge Graph (PKG):} $\mathcal{G}_{\mathrm{PKG}}=(\mathcal{N}_{\mathrm{PKG}},\mathcal{E}_{\mathrm{PKG}})$, derived from personalized embeddings $E_{\mathrm{sup}}(t,\xi)$.
  \item \textbf{Graph Fusion Operator:} $\oplus_{\mathcal{K}}:\mathcal{G}_{\mathrm{BKG}}\times\mathcal{G}_{\mathrm{PKG}}\to\mathcal{G}_{\mathrm{Unified}}$.
  \item \textbf{Unified Knowledge Graph:} $\mathcal{G}_{\mathrm{Unified}}=\mathcal{G}_{\mathrm{BKG}}\cup\mathcal{G}_{\mathrm{PKG}}$.
\end{itemize}

\subsubsection*{Non-Euclidean Knowledge Fusion}
Beyond simple set-union, knowledge integration is modeled in non-Euclidean spaces:
\[
\mathcal{G}_{\mathrm{Unified}}=\int_{\mathcal{M}_{\mathrm{geom}}}\exp_{\mathcal{G}_{\mathrm{BKG}}}\Bigl(t\,\mathcal{G}_{\mathrm{PKG}}\Bigr)\,dt,\quad t\in[0,1],
\]
reflecting continuous geometric transformations between graph domains.

\subsection{Infinite RoPE and Dynamic Vocabulary}

\begin{itemize}[label=\(\bullet\)]
  \item \textbf{RoPE Transformation:} $\psi:\mathbb{N}\times\mathbb{R}^{d_{\mathrm{att}}}\to\mathbb{R}^{d_{\mathrm{att}}}$.
  \item \textbf{Frequency Parameters:} $\theta^{\ast}_j$ for $j=1,\dots,\frac{d_{\mathrm{att}}}{2}$.
  \item \textbf{Dynamic Vocabulary Update:} $\Delta_{\mathcal{V}}:\mathcal{V}\times\mathcal{D}_{\mathrm{learn}}\to\mathcal{V}'$.
\end{itemize}

\subsubsection*{Hypergeometric Tokenization}
A novel tokenization scheme minimizes an energy function:
\[
\mathcal{T}_{\mathrm{base}}(X)=\argmin_{\{t_i\}}\sum_{k=1}^K\Biggl[\Re(z_k)-\sum_{j=1}^J\alpha_j\,\phi_j(t_k)\Biggr]^2+\lambda\,\Omega(\{\alpha_j\}),
\]
where $\phi_j$ are basis functions and $\Omega$ is a regularizer.

\subsection{Core Model Architecture}

\begin{itemize}[label=\(\bullet\)]
  \item \textbf{Deep Model:} $f_{\theta}:(\mathbb{R}^{d_F})^n\to\mathcal{Y}$, with parameters $\theta\in\Theta\subset\mathbb{R}^{p}$.
  \item \textbf{Transformer Sub-module:} $f^{\mathrm{T}}_{\theta_{\mathrm{T}}}$.
  \item \textbf{RWKV Sub-module:} $f^{\mathrm{RWKV}}_{\theta_{\mathrm{R}}}$, featuring recurrence variables $S_t$, $Z_t$ and a decay vector $\boldsymbol{\lambda}$.
  \item \textbf{Expert Coordinator:} $\Gamma:\{f^{(i)}_{\theta_i}\}_{i\in I_{\mathrm{exp}}}\to f^{\mathrm{Unified}}_{\theta}$.
\end{itemize}

\subsection{Titans Memory Architecture}

\begin{itemize}[label=\(\bullet\)]
  \item \textbf{Memory Bank:} $\mathcal{M}=\{(k^{\mathcal{M}}_i,v^{\mathcal{M}}_i)\}_{i=1}^{M_{\mathcal{M}}}$.
  \item \textbf{Similarity Function:} $s:\mathbb{R}^{d_L}\times\mathbb{R}^{d_k}\to\mathbb{R}$.
  \item \textbf{Attention Weights:}
  \[
    \alpha_i(x)=\frac{\exp\Bigl(s\bigl(r,k^{\mathcal{M}}_i\bigr)/\tau\Bigr)}
      {\sum_{j}\exp\Bigl(s\bigl(r,k^{\mathcal{M}}_j\bigr)/\tau\Bigr)}.
  \]
  \item \textbf{Aggregated Memory Read:} $m(x)=\sum_{i=1}^{M_{\mathcal{M}}}\alpha_i(x)\,v^{\mathcal{M}}_i$.
  \item \textbf{Meta-Learner:} $h:\mathbb{R}^{d_v}\to\mathbb{R}^{p}$, producing the parameter update $\Delta\theta(x)$.
\end{itemize}

\subsection{Recursive Adaptive Feedback}

\begin{itemize}[label=\(\bullet\)]
  \item \textbf{Recursive Entities:} $\mathcal{E}^{\mathrm{R}}=\{E^{\mathrm{R}}_i\}_{i\in I_{\mathrm{R}}}$.
  \item \textbf{Trigger and Action Functions:} $\tau^{\mathrm{R}}:\mathcal{E}^{\mathrm{R}}\to\mathcal{T}^{\mathrm{R}}$ and $\alpha^{\mathrm{R}}:\mathcal{E}^{\mathrm{R}}\to\mathcal{A}^{\mathrm{R}}$.
  \item \textbf{Influence Function:} $\Delta^{\mathrm{R}}:\mathcal{E}^{\mathrm{R}}\times\mathcal{E}^{\mathrm{R}}\to\mathcal{F}^{\mathrm{R}}$.
  \item \textbf{Feedback Composition:}
  \[
    \phi^{\mathrm{R}}\bigl(E^{\mathrm{R}}_i,E^{\mathrm{R}}_j\bigr)
    =\Delta^{\mathrm{R}}\bigl(E^{\mathrm{R}}_i,E^{\mathrm{R}}_j\bigr)
      \circ\beta^{\mathrm{R}}\bigl(E^{\mathrm{R}}_j,E^{\mathrm{R}}_i\bigr).
  \]
  \item \textbf{Overall Runtime State:} $\Sigma^{\mathrm{R}}$, updated via 
  \[
    \mu^{\mathrm{R}}:\Sigma^{\mathrm{R}}\to\Sigma^{\mathrm{R}},
  \]
  satisfying 
  \[
    \mu^{\mathrm{R}}\Bigl(\mu^{\mathrm{R}}(\Sigma^{\mathrm{R}})\Bigr)=\mu^{\mathrm{R}}(\Sigma^{\mathrm{R}}).
  \]
\end{itemize}

\subsection{Universal Training System}

\begin{itemize}[label=\(\bullet\)]
  \item \textbf{Loss Function:} $\mathcal{L}:\Theta\times\mathcal{D}\to\mathbb{R}_{\ge0}$, with mini-batch loss
  \[
    \mathcal{L}(\theta;B)=\frac{1}{|B|}\sum_{(x,y)\in B}\ell\Bigl(f_{\theta}(x),y\Bigr)
      +\lambda_W\,\|\theta\|^2+\lambda_{\mathrm{sparse}}\,\mathcal{R}_{\mathrm{sparse}}(\theta).
  \]
  \item \textbf{Optimizer:} $\mathcal{O}:\Theta\times\nabla_{\theta}\mathcal{L}\times\Xi\to\Theta$, where $\Xi$ represents the optimizer state (e.g., for AdamW).
  \item \textbf{Normalization Operator:} $N:\mathbb{R}^{d}\to\mathbb{R}^{d}$, defined as
  \[
    N(x)=\gamma\odot\frac{x-\mu_x}{\sqrt{\sigma_x^2+\epsilon}}+\beta.
  \]
  \item \textbf{Dropout Operator:} $D:\mathbb{R}^{d}\to\mathbb{R}^{d}$, using a Bernoulli mask $m\sim\mathrm{Bernoulli}(1-p)$.
  \item \textbf{Skip Connection:} $S(x,F(x))=x+F(x)$.
\end{itemize}

\subsection{Final Decoding and Multimodal Output}

\begin{itemize}[label=\(\bullet\)]
  \item \textbf{Decoding Function:} $\delta:\mathcal{Y}\to\mathcal{O}_{\mathrm{mod}}$, where $\mathcal{O}_{\mathrm{mod}}$ specifies the output modality (default: $\mathrm{Text}$).
  \item \textbf{Modality Decision Function:} $\mu_{\mathrm{mod}}:\mathcal{Y}\times\mathcal{C}_{\mathrm{task}}\to\mathcal{O}_{\mathrm{mod}}$, with $\mathcal{C}_{\mathrm{task}}$ representing task requirements.
\end{itemize}

\newpage
%%%%%%%%%%%%%%%%%%%%%%%%%%%%%%%%%%%%%%%%%%%%%%%%%%%%%%%%%%%%%%%%%%%%%%
\section{Enhanced Algorithmic Formalism}
%%%%%%%%%%%%%%%%%%%%%%%%%%%%%%%%%%%%%%%%%%%%%%%%%%%%%%%%%%%%%%%%%%%%%%

\begin{algorithm}
\caption{Eidos Quantum-Adaptive Inference with Module Annotations}
\begin{algorithmic}[1]
  \State \textbf{Input (Module A):} $\ket{X_{\mathrm{raw}}}=\bigotimes_{k=1}^K\ket{\sigma_k}$, entangled dataset $\mathcal{D}_{\mathrm{ent}}$
  \State \textbf{Initialize (Module I):} $\rho_0=\ket{0}\bra{0}^{\otimes n_{\mathrm{qvm}}}$
  \For{$t\in\mathbb{T}_{\mathrm{adapt}}$ \textbf{(Dynamic Adaptivity)}}
      \State $\ket{\psi_t}\gets\mathcal{F}_{\mathrm{rope}}\Bigl(\bigotimes_i\ket{z_i}\Bigr)$ \quad \textit{(Module G: Infinite RoPE)}
      \State Apply $\mathcal{U}_{\mathrm{mem}}=\exp\bigl(-i\mathcal{H}_{\mathcal{M}}t\bigr)$, where $\mathcal{H}_{\mathcal{M}}=\sum_i\alpha_i\,\mathcal{M}_i$ \quad \textit{(Module I: Memory Integration)}
      \State Measure: $\mathcal{G}_{\mathrm{Unified}}'\gets\Pi_{\mathrm{ctx}}\circ\mathcal{M}_{\mathrm{mem}}(\rho_t)$ \quad \textit{(Module F: Knowledge Graph Integration)}
      \State Update: $\theta_{t+1}\gets\theta_t\oplus\mathfrak{L}\bigl\{\mathcal{D}_{\mathrm{ent}},\mathcal{G}_{\mathrm{Unified}}'\bigr\}$ \quad \textit{(Module K: Training System)}
      \State $\rho_{t+1}\gets\mathcal{E}_{\mathrm{noise}}(\rho_t)\otimes\ket{\theta_{t+1}}\bra{\theta_{t+1}}$ \quad \textit{(Noise and Parameter Persistence)}
  \EndFor
  \State \textbf{Output (Module L):} $\Tr\Bigl(\rho_{\mathrm{final}}\Bigr)\otimes\mathcal{P}_{\mathrm{out}}$
\end{algorithmic}
\end{algorithm}

%%%%%%%%%%%%%%%%%%%%%%%%%%%%%%%%%%%%%%%%%%%%%%%%%%%%%%%%%%%%%%%%%%%%%%
\section{Complete Operator Taxonomy}
%%%%%%%%%%%%%%%%%%%%%%%%%%%%%%%%%%%%%%%%%%%%%%%%%%%%%%%%%%%%%%%%%%%%%%

\begin{table}[h]
\centering
\caption{Eidos Operator Hierarchy}
\begin{tabular}{lll}
\hline
\textbf{Symbol} & \textbf{Space} & \textbf{Properties} \\
\hline
$\mathfrak{E}$ & $\mathcal{L}(\mathbb{H}_{\mathrm{emb}})$ & Completely positive, trace-nonincreasing \\
$\mathcal{F}_{\mathrm{rope}}$ & $\mathbb{C}^{d_{\mathrm{att}}}$ & Rotational equivariance, $\mathfrak{so}(2n)$ representation \\
$\mathfrak{L}$ & $\Theta\times\mathcal{D}^{\infty}$ & Fr\'echet differentiable, $\nabla$-parallel \\
$\mathcal{U}_{\mathrm{mem}}$ & $\mathbb{H}_{\mathcal{M}}$ & Haar-random, ergodic \\
\hline
\end{tabular}
\end{table}

%%%%%%%%%%%%%%%%%%%%%%%%%%%%%%%%%%%%%%%%%%%%%%%%%%%%%%%%%%%%%%%%%%%%%%
\section{Complete Commutative Diagram}
%%%%%%%%%%%%%%%%%%%%%%%%%%%%%%%%%%%%%%%%%%%%%%%%%%%%%%%%%%%%%%%%%%%%%%

\[
\begin{tikzcd}
\Sigma^* \arrow[r, "\mathcal{P}_{\mathrm{in}}"] \arrow[d, "Entangle"'] & 
\mathcal{X}_{\mathrm{proc}} \arrow[r, "\mathcal{T}_{\mathrm{base}}"'] &
\mathcal{V}^{\otimes n} \arrow[d, "\mathfrak{E}"] \\
\mathbb{H}_{\mathrm{raw}} \arrow[r, "\mathcal{U}_{\mathrm{prep}}"'] &
\mathbb{H}_{\mathrm{emb}} \arrow[r, "\mathcal{F}_{\mathrm{rope}}"'] &
\mathbb{H}_{\mathrm{ctx}} \arrow[d, "\mathcal{M}_{\mathrm{mem}}"] \\
& \mathbb{H}_{\mathcal{M}} \arrow[r, "\mathfrak{L}"'] &
\Theta_{\mathrm{adapt}} \arrow[u, "\mathcal{E}_{\mathrm{fb}}"']
\end{tikzcd}
\]

%%%%%%%%%%%%%%%%%%%%%%%%%%%%%%%%%%%%%%%%%%%%%%%%%%%%%%%%%%%%%%%%%%%%%%
\section{Formal Verification}
%%%%%%%%%%%%%%%%%%%%%%%%%%%%%%%%%%%%%%%%%%%%%%%%%%%%%%%%%%%%%%%%%%%%%%

\subsection*{Consistency Proof}
\begin{proof}[Consistency of the Recursive Feedback]
Assume $\mu^{\mathrm{R}}: \Sigma^{\mathrm{R}} \to \Sigma^{\mathrm{R}}$ is a contraction mapping under the norm $\|\cdot\|$. Then, by the Banach Fixed-Point Theorem,
\[
\|\mu^{\mathrm{R}}(\Sigma)-\mu^{\mathrm{R}}(\Sigma')\|\le L\|\Sigma-\Sigma'\|,\quad \text{with } L<1.
\]
Since the idempotence axiom (A2) holds, we have $L=0$, guaranteeing the existence of a unique fixed point $\Sigma_{\mathrm{fix}}$.
\end{proof}

\begin{proof}
Under the contraction mapping assumption on $\mu^{\mathrm{R}}$, the Banach Fixed-Point Theorem implies a unique fixed point exists. Given the idempotence condition,
\[
\mu^{\mathrm{R}}(\mu^{\mathrm{R}}(\Sigma))=\mu^{\mathrm{R}}(\Sigma),
\]
conclude that $\Sigma_{\mathrm{fix}}=\mu^{\mathrm{R}}(\Sigma_{\mathrm{fix}})$ is unique.
\end{proof}

\subsection*{Universal Approximation Property}
\begin{proof}[Universal Approximation]
For any Borel measure $\nu$ defined on $(\R^{d_F},\mathcal{B})$, there exists a parameter vector $\theta^*\in\Theta$ satisfying
\[
d_{\mathrm{TV}}\Bigl(f_{\theta^*}(\mathcal{X}),\nu\Bigr)<\epsilon+\lambda_{\mathrm{sparse}}\|\theta^*\|_{\ell^0},
\]
where $d_{\mathrm{TV}}$ denotes the total variation distance.
\end{proof}

\begin{proof}
This follows from the $\Gamma$-density property of the mixture-of-experts in the function space $\mathcal{C}(\R^{d_F},\mathcal{Y})$. By suitably choosing the expert parameters and sparse regularization, the target distribution $\nu$ can be approximated arbitrarily closely.
\end{proof}

%%%%%%%%%%%%%%%%%%%%%%%%%%%%%%%%%%%%%%%%%%%%%%%%%%%%%%%%%%%%%%%%%%%%%%
\section{End-to-End Data Flow and Processing Sequence}
%%%%%%%%%%%%%%%%%%%%%%%%%%%%%%%%%%%%%%%%%%%%%%%%%%%%%%%%%%%%%%%%%%%%%%

This section delineates the complete data and model flow within the Eidos framework.

\subsection*{Step 1: Input Processing (Module A)}
\begin{itemize}[label=\(\bullet\)]
  \item \textbf{Raw Input:} $X_{\mathrm{raw}}\in\Sigma^*$ (e.g., a text document).
  \item \textbf{Preprocessing:} Compute $X_{\mathrm{proc}}\gets\mathcal{P}_{\mathrm{in}}(X_{\mathrm{raw}})$, where $X_{\mathrm{proc}}\in\mathcal{X}_{\mathrm{proc}}$.
\end{itemize}

\subsection*{Step 2: Universal Communication and Data Handling (Module B)}
\begin{itemize}[label=\(\bullet\)]
  \item Encapsulate each message as a universal data packet
  \[
    \mathcal{P}=\Bigl(\operatorname{ID}_{\mathcal{P}},\,\mathrm{Payload}_{\mathcal{P}},\,\mathrm{Meta}_{\mathcal{P}}\Bigr).
  \]
  \item Utilize communication channels $\mathcal{C}_{ij}$ with routing function $\mathcal{R}_{\mathrm{comm}}$ to orchestrate module interactions.
  \item The coordination manager $\Omega$ registers modules and ensures reliable packet delivery.
\end{itemize}

\subsection*{Step 3: Universal Streaming, Loading, and Chunking (Module C)}
\begin{itemize}[label=\(\bullet\)]
  \item Partition model parameters as $\theta=\bigcup_{j\in J} T_j$.
  \item Index each chunk $T_j$ by $\mathcal{I}(T_j)=\ell_j\in\mathcal{S}_{\mathrm{disk}}$.
  \item Dynamically stream chunks into memory via $\sigma$ and cache them using $\mu$.
\end{itemize}

\subsection*{Step 4: Multidimensional Vocabulary and Tokenization (Module D)}
\begin{itemize}[label=\(\bullet\)]
  \item Define the complete vocabulary as $\mathcal{V}=\mathcal{V}^{(0)}\cup\mathcal{V}^{(1)}$.
  \item Assign each token $t\in\mathcal{V}$ a unique identifier, $\operatorname{ID}(t)=\eta(t)$.
  \item Segment the preprocessed input $X_{\mathrm{proc}}$ into tokens $(t_1,\dots,t_n)$ using the base tokenizer $\mathcal{T}_{\mathrm{base}}$.
\end{itemize}

\subsection*{Step 5: Contextual Embedding and Tokenization (Module E)}
\begin{itemize}[label=\(\bullet\)]
  \item For each token $t_i$, compute its base embedding $\mathbf{e}_i=E_{\mathrm{B}}(t_i)\in\mathbb{R}^{d_E}$.
  \item Process the sequence $(\mathbf{c}_1,\dots,\mathbf{c}_n)=E_{\mathrm{C}}(\mathbf{e}_1,\dots,\mathbf{e}_n)$.
  \item Fuse base and contextual embeddings: $\mathbf{z}_i=g(\mathbf{e}_i,\mathbf{c}_i)\in\mathbb{R}^{d_F}$.
  \item Integrate dual-layer representations via
  \[
    \mathbf{z}_i=g\Bigl(E_{\mathrm{B}}(t_i),\,E_{\mathrm{sup}}(t_i,\xi)\Bigr).
  \]
\end{itemize}

\subsection*{Step 6: Deep Knowledge Graph Construction (Module F)}
\begin{itemize}[label=\(\bullet\)]
  \item Build the Base Knowledge Graph:
  \[
    \mathcal{G}_{\mathrm{BKG}}=\Bigl(\mathcal{N}_{\mathrm{BKG}},\,\mathcal{E}_{\mathrm{BKG}}\Bigr),
  \]
  where nodes represent tokens $t$ with embeddings $E_{\mathrm{B}}(t)$ and edges are defined by $\rho_{\mathrm{base}}(t_i,t_j)\subset\mathcal{R}_{\mathrm{base}}$.
  \item Construct the Personal Knowledge Graph:
  \[
    \mathcal{G}_{\mathrm{PKG}}=\Bigl(\mathcal{N}_{\mathrm{PKG}},\,\mathcal{E}_{\mathrm{PKG}}\Bigr),
  \]
  utilizing personalized embeddings $E_{\mathrm{sup}}(t,\xi)$.
  \item Fuse the graphs via $\oplus_{\mathcal{K}}$ to obtain
  \[
    \mathcal{G}_{\mathrm{Unified}}=\mathcal{G}_{\mathrm{BKG}}\cup\mathcal{G}_{\mathrm{PKG}}.
  \]
\end{itemize}

\subsection*{Step 7: Infinite RoPE and Dynamic Vocabulary Updating (Module G)}
\begin{itemize}[label=\(\bullet\)]
  \item For each token position $i$ and each 2D subspace $j$, define the rotation matrix
  \[
    R^{(j)}(i)=\begin{pmatrix}
      \cos\bigl(i\,\theta^{\ast}_j\bigr) & -\sin\bigl(i\,\theta^{\ast}_j\bigr)\\[1mm]
      \sin\bigl(i\,\theta^{\ast}_j\bigr) & \cos\bigl(i\,\theta^{\ast}_j\bigr)
    \end{pmatrix}.
  \]
  \item Form the block-diagonal rotation:
  \[
    R(i)=\operatorname{diag}\Bigl(R^{(1)}(i),\dots,R^{(d_{\mathrm{att}}/2)}(i)\Bigr),\quad\psi(i,v)=R(i)v.
  \]
  \item Apply $\psi(i,\cdot)$ to the query/key vectors in the attention mechanism.
  \item Dynamically integrate newly learned tokens via 
  \[
    \Delta_{\mathcal{V}}:\mathcal{V}\times\mathcal{D}_{\mathrm{learn}}\to\mathcal{V}'.
  \]
\end{itemize}

\subsection*{Step 8: Core Model Processing (Module H)}
\begin{itemize}[label=\(\bullet\)]
  \item Process the fused embeddings through the deep model:
  \[
    f_{\theta}:(\mathbb{R}^{d_F})^n\to\mathcal{Y},\quad\theta\in\Theta.
  \]
  \item Sub-modules include:
  \begin{itemize}[label=\(\circ\)]
    \item \textbf{Transformer:} $f^{\mathrm{T}}_{\theta_{\mathrm{T}}}$, leveraging multi-head self-attention (enhanced with RoPE).
    \item \textbf{RWKV:} $f^{\mathrm{RWKV}}_{\theta_{\mathrm{R}}}$ with recurrence defined as
    \[
      S_t=\boldsymbol{\lambda}\odot S_{t-1}+\exp(k_t)\odot v_t,\quad
      Z_t=\boldsymbol{\lambda}\odot Z_{t-1}+\exp(k_t).
    \]
  \end{itemize}
  \item Coordinate the expert models via
  \[
    f^{\mathrm{Unified}}_{\theta}=\Gamma\Bigl(\{f^{(i)}_{\theta_i}\}_{i\in I_{\mathrm{exp}}}\Bigr).
  \]
\end{itemize}

\subsection*{Step 9: Titans Memory Architecture (Module I)}
\begin{itemize}[label=\(\bullet\)]
  \item Define the memory bank:
  \[
    \mathcal{M}=\{(k^{\mathcal{M}}_i,v^{\mathcal{M}}_i)\}_{i=1}^{M_{\mathcal{M}}}.
  \]
  \item For a latent representation $r$, compute similarities $s\bigl(r,k^{\mathcal{M}}_i\bigr)$ and attention weights
  \[
    \alpha_i(x)=\frac{\exp\Bigl(s\bigl(r,k^{\mathcal{M}}_i\bigr)/\tau\Bigr)}
      {\sum_{j}\exp\Bigl(s\bigl(r,k^{\mathcal{M}}_j\bigr)/\tau\Bigr)}.
  \]
  \item Aggregate the memory read:
  \[
    m(x)=\sum_{i=1}^{M_{\mathcal{M}}}\alpha_i(x)v^{\mathcal{M}}_i.
  \]
  \item Obtain the parameter update via the meta-learner:
  \[
    \Delta\theta(x)=h\bigl(m(x)\bigr),\quad\theta_x=\theta+\Delta\theta(x).
  \]
\end{itemize}

\subsection*{Step 10: Recursive Adaptive Feedback (Module J)}
\begin{itemize}[label=\(\bullet\)]
  \item Define runtime recursive entities: $\mathcal{E}^{\mathrm{R}}=\{E^{\mathrm{R}}_i\}$.
  \item Construct feedback using trigger $\tau^{\mathrm{R}}$, action $\alpha^{\mathrm{R}}$, and influence $\Delta^{\mathrm{R}}$:
  \[
    \phi^{\mathrm{R}}\bigl(E^{\mathrm{R}}_i,E^{\mathrm{R}}_j\bigr)
    =\Delta^{\mathrm{R}}\bigl(E^{\mathrm{R}}_i,E^{\mathrm{R}}_j\bigr)
      \circ\beta^{\mathrm{R}}\bigl(E^{\mathrm{R}}_j,E^{\mathrm{R}}_i\bigr).
  \]
  \item Update the overall state: $\Sigma^{\mathrm{R}}\gets\mu^{\mathrm{R}}(\Sigma^{\mathrm{R}})$ (ensuring idempotence).
\end{itemize}

\subsection*{Step 11: Universal Training System (Module K)}
\begin{itemize}[label=\(\bullet\)]
  \item For a mini-batch $B\subset\mathcal{D}$, compute the loss:
  \[
    \mathcal{L}(\theta;B)=\frac{1}{|B|}\sum_{(x,y)\in B}\ell\Bigl(f_{\theta}(x),y\Bigr)
      +\lambda_W\,\|\theta\|^2+\lambda_{\mathrm{sparse}}\,\mathcal{R}_{\mathrm{sparse}}(\theta).
  \]
  \item Update parameter chunks via the optimizer:
  \[
    T_j\leftarrow\mathcal{O}\Bigl(T_j,\nabla_{T_j}\mathcal{L},\Xi_j\Bigr).
  \]
  \item Within the forward pass, apply normalization $N$, dropout $D$, and skip connections $S$.
  \item Continuously stream and commit parameter updates using $\sigma$.
\end{itemize}

\subsection*{Step 12: Final Decoding and Multimodal Output (Module L)}
\begin{itemize}[label=\(\bullet\)]
  \item The deep model produces a latent output $y_{\mathrm{latent}}\in\mathcal{Y}$.
  \item Decode the latent representation using $\delta:\mathcal{Y}\to\mathrm{Text}$ to obtain $\hat{y}_{\mathrm{text}}=\delta\bigl(y_{\mathrm{latent}}\bigr)$.
  \item Use $\mu_{\mathrm{mod}}:\mathcal{Y}\times\mathcal{C}_{\mathrm{task}}\to\mathcal{O}_{\mathrm{mod}}$ to determine any additional modalities (e.g., images, audio).
  \item Package the final output as 
  \[
    \mathcal{P}_{\mathrm{out}}=\Bigl(\operatorname{ID}_{\mathrm{out}},\hat{y},\mathrm{Meta}_{\mathrm{out}}\Bigr),
  \]
  and route it via $\Omega$ for logging and feedback.
\end{itemize}

\newpage
%%%%%%%%%%%%%%%%%%%%%%%%%%%%%%%%%%%%%%%%%%%%%%%%%%%%%%%%%%%%%%%%%%%%%%
\section{Integrated End-to-End Algorithm}
%%%%%%%%%%%%%%%%%%%%%%%%%%%%%%%%%%%%%%%%%%%%%%%%%%%%%%%%%%%%%%%%%%%%%%

\begin{algorithm}
\caption{Eidos Integrated Inference and Training Pipeline}
\label{alg:eidos}
\begin{algorithmic}[1]
  \State \textbf{Input:} Raw input $X_{\mathrm{raw}}\in\Sigma^*$, training set $\mathcal{D}$, task context $\mathcal{C}_{\mathrm{task}}$
  \State \textbf{Preprocess:} $X_{\mathrm{proc}}\gets\mathcal{P}_{\mathrm{in}}(X_{\mathrm{raw}})$
  \State \textbf{Tokenize:} $(t_1,\dots,t_n)\gets\mathcal{T}_{\mathrm{base}}(X_{\mathrm{proc}})$
  \For{$i=1$ to $n$}
      \State $\mathbf{e}_i\gets E_{\mathrm{B}}(t_i)$
  \EndFor
  \State $(\mathbf{c}_1,\dots,\mathbf{c}_n)\gets E_{\mathrm{C}}(\mathbf{e}_1,\dots,\mathbf{e}_n)$
  \For{$i=1$ to $n$}
      \State $\mathbf{z}_i\gets g(\mathbf{e}_i,\mathbf{c}_i)$
      \State $\mathbf{z}_i'\gets\psi(i,\mathbf{z}_i)$
  \EndFor
  \State \textbf{Knowledge Graph:} Update $\mathcal{G}_{\mathrm{BKG}}$ and $\mathcal{G}_{\mathrm{PKG}}$, compute 
  \[
    \mathcal{G}_{\mathrm{Unified}}\gets\mathcal{G}_{\mathrm{BKG}}\cup\mathcal{G}_{\mathrm{PKG}}.
  \]
  \State $y_{\mathrm{latent}}\gets f_{\theta}(\mathbf{z}_1',\dots,\mathbf{z}_n')$
  \State \textbf{Test-Time Adaptation:}
  \State $r\gets g_{\theta}(X_{\mathrm{proc}})$
  \For{$i=1$ to $M_{\mathcal{M}}$}
      \State $s_i\gets s\bigl(r,k^{\mathcal{M}}_i\bigr)$
  \EndFor
  \State Compute $\alpha_i\gets\frac{\exp(s_i/\tau)}{\sum_{j}\exp(s_j/\tau)}$, and $m(x)\gets\sum_i\alpha_i\,v^{\mathcal{M}}_i$
  \State $\Delta\theta(x)\gets h\bigl(m(x)\bigr),\quad\theta_x\gets\theta+\Delta\theta(x)$
  \State \textbf{Recursive Feedback:} $\Sigma^{\mathrm{R}}\gets\mu^{\mathrm{R}}(\Sigma^{\mathrm{R}})$
  \State \textbf{Decoding:} $\hat{y}\gets\delta\bigl(y_{\mathrm{latent}}\bigr)$
  \If{multimodal output needed}
      \State $\hat{y}_{\mathrm{mod}}\gets\mu_{\mathrm{mod}}(y_{\mathrm{latent}},\mathcal{C}_{\mathrm{task}})$
  \EndIf
  \State $\mathcal{P}_{\mathrm{out}}\gets\Bigl(\operatorname{ID}_{\mathrm{out}},\hat{y},\mathrm{Meta}_{\mathrm{out}}\Bigr)$
  \If{training mode}
      \State \textbf{/* Execute training loop (refer to supplementary Algorithm~\ref{alg:training}) */}
  \EndIf
\end{algorithmic}
\end{algorithm}

\newpage
%%%%%%%%%%%%%%%%%%%%%%%%%%%%%%%%%%%%%%%%%%%%%%%%%%%%%%%%%%%%%%%%%%%%%%
\section{Conclusion}
%%%%%%%%%%%%%%%%%%%%%%%%%%%%%%%%%%%%%%%%%%%%%%%%%%%%%%%%%%%%%%%%%%%%%%

We have presented the \textbf{Eidos} framework with exhaustive technical definitions and rigorous mathematical formalism, ensuring that every component is fully integrated and systematically defined. The document now features:

\begin{enumerate}[label=(\Alph*)]
  \item \textbf{Input Processing:} The raw input $X_{\mathrm{raw}}$ is transformed into a processed form $X_{\mathrm{proc}}$ via the operator $\mathcal{P}_{\mathrm{in}}$, with precise domain and range definitions.
  \item \textbf{Universal Communication and Data Handling:} Data packets $\mathcal{P}$ and robust communication channels $\mathcal{C}_{ij}$ are employed, with the coordination manager $\Omega$ ensuring integration across modules.
  \item \textbf{Streaming, Loading, and Chunking:} Model parameters $\theta$, partitioned into chunks $\{T_j\}$, are systematically streamed and cached, ensuring consistent updates.
  \item \textbf{Multidimensional Vocabulary and Tokenization:} A comprehensive vocabulary $\mathcal{V}$ is structured with unique identifiers $\eta$, and tokenization via $\mathcal{T}_{\mathrm{base}}$ is robustly defined.
  \item \textbf{Contextual Embedding and Tokenization:} Dual-layer representations are obtained by fusing $E_{\mathrm{B}}$ and $E_{\mathrm{C}}$ with the fusion operator $g$, integrating enhanced adaptability through $E_{\mathrm{sup}}$.
  \item \textbf{Knowledge Graphs:} Graphs $\mathcal{G}_{\mathrm{BKG}}$ and $\mathcal{G}_{\mathrm{PKG}}$ are meticulously constructed and fused into $\mathcal{G}_{\mathrm{Unified}}$, with non-Euclidean geometric integration.
  \item \textbf{Infinite RoPE and Dynamic Vocabulary:} The RoPE operator $\psi$ and dynamic update operator $\Delta_{\mathcal{V}}$ ensure that positional encoding and vocabulary refinement occur seamlessly.
  \item \textbf{Core Model Architectures:} The deep model $f_{\theta}$ integrates both Transformer ($f^{\mathrm{T}}_{\theta_{\mathrm{T}}}$) and RWKV ($f^{\mathrm{RWKV}}_{\theta_{\mathrm{R}}}$) sub-modules, coordinated by $\Gamma$, for a unified inference process.
  \item \textbf{Titans Memory Architecture:} Memory bank $\mathcal{M}$, with similarity measures and meta-learner $h$, provides adaptive parameter updates $\Delta\theta(x)$.
  \item \textbf{Recursive Adaptive Feedback:} The operator $\mu^{\mathrm{R}}$ enforces idempotence and systematic integration of recursive feedback.
  \item \textbf{Universal Training System:} Comprehensive loss functions $\mathcal{L}$, coupled with optimizers $\mathcal{O}$, and auxiliary operators $N$, $D$, and $S$, form a robust training loop.
  \item \textbf{Final Decoding and Multimodal Output:} The function $\delta$ and modality decision operator $\mu_{\mathrm{mod}}$ ensure that the final output is delivered in the desired format, with extensibility to other modalities.
\end{enumerate}

This rigorously defined and fully integrated framework offers a reliable blueprint for constructing a persistent, adaptive, and multimodal intelligent system, ensuring every aspect is consistent and systematically utilized from the input stage to the final output.

\end{document}

%%%%%%%%%%%%%%%%%%%%%%%%%%%%%%%%%%%%%%%%%%%%%%%%%%%%%%%%%%%%%%%%%%%%%%
% End perfected file: eidos_complete.tex
%%%%%%%%%%%%%%%%%%%%%%%%%%%%%%%%%%%%%%%%%%%%%%%%%%%%%%%%%%%%%%%%%%%%%%

The above revised code fixes every reported chktex issue while preserving every detail exactly as in your original document (with enhanced typographic and stylistic quality)
